% Documentation avec latex2man
%
% Auteur  : Gregory DAVID
%%

\documentclass[french]{article}
\usepackage[utf8]{inputenc}
\usepackage[french]{babel}
\usepackage{latex2man}

%% avons-nous le package 'gitinfo' ?
\IfFileExists{gitinfo.sty}{
  \usepackage{gitinfo}
  \setDate{\gitAuthorIsoDate}
  \setVersion{\gitAbbrevHash}
}{
  %%%% sinon on definit la date a la main
  \setDate{\today}
  \setVersion{0}
}

\begin{document}

\begin{Name}{1}{playlistGenerator}{Grégory DAVID}{Playlist generator}{A simple playlist generator}

    \Prog{playlistGenerator} is a playlist generator tool based on a
    remote musical database. You can specify proportions, according to:
    \begin{itemize}
        \item genre
        \item sub-genre
        \item track title
        \item album name
        \item band or artist name
    \end{itemize}
\end{Name}

\section{Synopsis}
%%%%%%%%%%%%%%%%%%
\Prog{playlistGenerator}
 \oOpt{-h}\\
 \SP\oOptArg{--log }{\{DEBUG,INFO,WARNING,ERROR,CRITICAL\}}\\
 \SP\oOptArg{--output }{\{m3u,xspf,pls\} FILENAME} \\
 \SP\oOptArg{(-g\Bar--genre) }{GENRE QUANTITY}\\
 \SP\oOptArg{(-s\Bar--sub-genre) }{SUB\_GENRE QUANTITY}\\
 \SP\oOptArg{(-b\Bar--band) }{BAND\_NAME QUANTITY}\\
 \SP\oOptArg{(-a\Bar--album) }{ALBUM QUANTITY}\\
 \SP\oOptArg{(-t\Bar--title) }{TRACK\_TITLE QUANTITY}\\
 \SP\oOptArg{(-G\Bar--RE-genre) }{GENRE QUANTITY}\\
 \SP\oOptArg{(-S\Bar--RE-sub-genre) }{SUB\_GENRE QUANTITY}\\
 \SP\oOptArg{(-B\Bar--RE-band) }{BAND\_NAME QUANTITY}\\
 \SP\oOptArg{(-A\Bar--RE-album) }{ALBUM QUANTITY}\\
 \SP\oOptArg{(-T\Bar--RE-title) }{TRACK\_TITLE QUANTITY}\\

\section{Description}
%%%%%%%%%%%%%%%%%%%%%
\Prog{playlistGenerator} aimed to generate an outputed playlist file
(i.e.: \File{\Tilde/path/to/my/playlist}) according to the user
selected output format.

Options ares described in section OPTIONS.

\section{Options}
%%%%%%%%%%%%%%%%%
%   --time LENGTH_IN_MINUTES
%                         Total playlist length, in minutes
%   --log {DEBUG,INFO,WARNING,ERROR,CRITICAL}
%   --output {m3u,xspf,pls} FILENAME
%                         Output format {m3u,xspf,pls} followed by filename
%                         (absolute or relative path, or '-' for stdout)

\begin{Description}\setlength{\itemsep}{0cm}
    \item[\Opt{-h}\Bar\Opt{--help}] show this help message and exit
    % \item[\Opt{-V}] Affiche les informations de version.

    \item[\OptArg{--time }{LENGTH\_IN\_MINUTES}] Total playlist
    length, in minutes
    \item[\OptArg{--log }{DEBUG\Bar INFO\Bar WARNING\Bar ERROR\Bar CRITICAL}]
    Defines the log level


    \item[\Opt{-g}\Bar\Opt{--genre} \Arg{GENRE} \Arg{QUANTITY}] genre
    to be included to playlist, followed by the \% quantity (see
    RULES)
    \item[\Opt{-s}\Bar\Opt{--sub-genre} \Arg{SUB\_GENRE} \Arg{QUANTITY}]
    sub-genre to be included to playlist, followed by
    the \% quantity (see RULES)
    \item[\Opt{-b}\Bar\Opt{--band} \Arg{BAND\_NAME} \Arg{QUANTITY}]
    band to be included to playlist, followed by the \% quantity (see
    RULES)
    \item[\Opt{-a}\Bar\Opt{--album} \Arg{ALBUM} \Arg{QUANTITY}] album
    to be included to playlist, followed by the \% quantity (see
    RULES)
    \item[\Opt{-t}\Bar\Opt{--title} \Arg{TRACK\_TITLE} \Arg{QUANTITY}]
    title to be included to playlist, followed by the \% quantity (see
    RULES)

    \item[\Opt{-G}\Bar\Opt{--RE-genre} \Arg{GENRE} \Arg{QUANTITY}]
    Regular Expression defining a genre to be included to playlist,
    followed by the \% quantity (see RULES, REGEXP)
    \item[\Opt{-S}\Bar\Opt{--RE-sub-genre} \Arg{SUB\_GENRE} \Arg{QUANTITY}]
    Regular Expression defining a sub-genre to be
    included to playlist, followed by the \% quantity (see RULES, REGEXP)
    \item[\Opt{-B}\Bar\Opt{--RE-band} \Arg{BAND\_NAME} \Arg{QUANTITY}]
    Regular Expression defining a band to be included to playlist,
    followed by the \% quantity (see RULES, REGEXP)
    \item[\Opt{-A}\Bar\Opt{--RE-album} \Arg{ALBUM} \Arg{QUANTITY}]
    Regular Expression defining a album to be included to playlist,
    followed by the \% quantity (see RULES, REGEXP)
    \item[\Opt{-T}\Bar\Opt{--RE-title} \Arg{TRACK\_TITLE} \Arg{QUANTITY}]
    Regular Expression defining a title to be included
    to playlist, followed by the \% quantity (see RULES, REGEXP)
\end{Description}


\section{Rules}
%%%%%%%%%%%%%%

The quantity defined the \% of the corresponding option to be included
in the output playlist. The quantity rules are:
\begin{enumerate}
    \item must be an integer, else an error is produced and
    \Prog{playlistGenerator} exits
    \item if the value is negative, absolute value is kept and
    \Prog{playlistGenerator} continues
    \item if the value is greater than 100\%, the value is replaced by
    the best value fitting the length left space
\end{enumerate}


\section{RegExp}
%%%%%%%%%%%%%%%

If you want to use RegExp, you have to follow the Python \Prog{re}
module implementation:
\URL{https://docs.python.org/3.2/library/re.html}

An error inside Python \Prog{re} module will result in a logged exit

\section{Files}
%%%%%%%%%%%%%%%

\begin{Description}\setlength{\itemsep}{0cm}
    \item[\File{/usr/local/bin/playlistGenerator}] Program file
    \item[\File{/var/log/playlistGenerator.log}] log file
\end{Description}

\section{See also}
%%%%%%%%%%%%%%%%%%

\Cmd{vlc}{1}, \Cmd{mp3rename}{1}.

\section{Translation}
%%%%%%%%%%%%%%%%%%%%%%%%%%%%%%%%%%%%%%%%%%%%%%%

Originaly written in English, we need translations to foreign languages:
\begin{itemize}
    \item German
    \item Italian
    \item Spanish
    \item Russian
    \item Chinese
    \item Japanese
\end{itemize}

\section{Requirements}
%%%%%%%%%%%%%%%%%%%%%%

\begin{description}\setlength{\itemsep}{0cm}
    \item[python] $>=$ 3.2
    \item[sqlalchemy] $>=$ 0.9.7, to access remote database
\end{description}

\section{Version}
%%%%%%%%%%%%%%%%%

Version : git@\Version\ du \Date.

\section{License and Copyright}
%%%%%%%%%%%%%%%%%%%%%%%%%%%%%%%

\begin{description}
    \item[Copyright] \copyright\ 2012, Grégory DAVID, 3 rue Beau
    Soleil, 72700 Allonnes
    \item[Licence] GPLv3.

    \item[Misc] If you find this software useful, please send me a
    postcard.
\end{description}

\section{Author}
%%%%%%%%%%%%%%%%

\noindent
Grégory DAVID                      \\
3 rue Beau Soleil                       \\
72700 Allonnes                       \\
Email: \Email{gregory.david@ac-nantes.fr}  \\
WWW: \URL{http://www.bts-malraux72.net}.\\

\LatexManEnd

\end{document}
